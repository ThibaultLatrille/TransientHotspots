\documentclass[8pt]{beamer}

\usepackage{amssymb,amsfonts,amsmath,amsthm,mathtools}
\usepackage{lmodern}
\usepackage{xfrac, nicefrac}
\usepackage{pgfplots, pgf,tikz}
\usepackage{enumitem}
\usepgfplotslibrary{fillbetween}
\usebackgroundtemplate{\tikz\node[opacity=0]{};}
\setbeamertemplate{footline}[frame number]{}
\setbeamertemplate{navigation symbols}{}
\setbeamertemplate{footline}{}
\usefonttheme{serif}
\pgfplotsset{compat=1.16}

\definecolor{RED}{HTML}{EB6231}
\definecolor{YELLOW}{HTML}{E29D26}
\definecolor{BLUE}{HTML}{5D80B4}
\definecolor{LIGHTGREEN}{HTML}{6ABD9B}
\definecolor{GREEN}{HTML}{8FB03E}
\definecolor{PURPLE}{HTML}{BE1E2D}
\definecolor{BROWN}{HTML}{A97C50}
\definecolor{PINK}{HTML}{DA1C5C}

\pgfplotsset{every axis/.append style={line width=1pt}}
\pgfplotscreateplotcyclelist{colors}{LIGHTGREEN\\YELLOW\\RED\\GREEN\\BLUE\\}


\newcommand{\der}{\mathrm{d}}
\newcommand{\e}{\text{e}}
\newcommand{\Ne}{N_{\text{e}}}
\newcommand{\pnps}{\pn / \ps}
\newcommand{\proba}{\mathbb{P}}
\newcommand{\pfix}{\proba_{\text{fix}}}

\newcommand{\ci}{{\color{BLUE}{\textbf{ATT}}}}
\newcommand{\cj}{{\color{YELLOW}\textbf{ATG}}}
\newcommand{\nuci}{{\color{BLUE}\textbf{T}}}
\newcommand{\nucj}{{\color{YELLOW}\textbf{G}}}
\newcommand{\aai}{{\color{BLUE}\textbf{Ile}}}
\newcommand{\aaj}{{\color{YELLOW}\textbf{Met}}}
\newcommand{\Fi}{{F_{\aai}}}
\newcommand{\Fj}{{F_{\aaj}}}
\newcommand{\aaitoj}{{\aai \rightarrow \aaj}}
\newcommand{\nucitoj}{\nuci \rightarrow \nucj}
\newcommand{\citoj}{\ci \rightarrow \cj}
\newcommand{\AtoB}{A \rightarrow B}
\newcommand{\itoj}{ i \rightarrow j }

\begin{document}
	\begin{frame}
		\begin{itemize}[label=$\bullet$]
		\item Subset of mutations: from weak (AT) to strong (GC).
		\item Under gBGC with conversion rate $b$ (for heterozygous).
		\item Mutation are deleterious with selection coefficient $s$ (homozygous) and $hs$ (heterozygous).
		\item $x_{t}$ is the frequency of the allele at time $t$.
		\end{itemize}
		\begin{equation*}
			\begin{dcases}
				x_{t+1} & \sim \mathcal{B} \left( \Ne, x_{t}' \right)  \\
				x_{t}' & = (1 - s)x_{t}^2 + (1 + b)(1 - hs)x_{t} (1 - x_{t}) \\
			\end{dcases}
		\end{equation*}
	\end{frame}
	\begin{frame}
		\centering
		\begin{tikzpicture}
			\begin{axis}[
				width=\textwidth,
				height=0.5\textwidth,
				ylabel={Relative fixation probability},
				xlabel={Scaled selection coefficient ($S=4 \Ne s$)},
				cycle list name=colors,
				domain=-5:5,
				ymin=0.0, ymax=5.0,
				samples=200,
				legend entries={$\frac{S}{1 - \e^{-S}}$},
				legend cell align=left,
				minor tick num=2,
				axis x line=bottom,
				axis y line=left,
				legend style={at={(0.02,0.9)},anchor=north west}
				]
				\addplot[line width=2.0pt, BLUE]{ x / (1 - exp(- x))};
				\addplot[name path=B, dashed, YELLOW, line width=0.5pt] coordinates {(-1, 0) (-1, 5)};
				\addplot[name path=A, line width=0pt] coordinates {(-5, 0) (-5, 5)};
				\addplot[black, dashed, line width=1.0pt]{1.0};
				\addplot[black, dashed, line width=1.0pt] coordinates {(0, 0) (0, 5)};
				\addplot[name path=C, dashed, YELLOW, line width=0.5pt] coordinates {(1, 0) (1, 5)};
				\addplot[name path=D, line width=0pt] coordinates {(5, 0) (5, 5)};
				\addplot[fill=RED, opacity=0.2] fill between[ of = A and B];
				\addplot[fill=YELLOW, opacity=0.2] fill between[ of = B and C];
				\addplot[fill=GREEN, opacity=0.2] fill between[ of = C and D];
			\end{axis}
		\end{tikzpicture}
	\end{frame}
\end{document}