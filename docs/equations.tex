\documentclass[8pt,aspectratio=169]{beamer}

\usepackage{amssymb,amsfonts,amsmath,amsthm,mathtools}
\usepackage{lmodern}
\usepackage{xfrac, nicefrac}
\usepackage{pgfplots, pgf,tikz}
\usepackage{enumitem}
\usepackage{tabu}
\usepackage{adjustbox}
\usepackage{bm}
\usepackage{wasysym}
\usepgfplotslibrary{fillbetween}
\usebackgroundtemplate{\tikz\node[opacity=0]{};}
\setbeamertemplate{footline}[frame number]{}
\setbeamertemplate{navigation symbols}{}
\setbeamertemplate{footline}{}
\usefonttheme{serif}
\pgfplotsset{compat=1.16}

\newcommand{\der}{\mathrm{d}}
\newcommand{\e}{\text{e}}
\newcommand{\Ne}{N_{\text{e}}}
\newcommand{\pnps}{\pn / \ps}
\newcommand{\proba}{\mathbb{P}}
\newcommand{\pfix}{\proba_{\text{fix}}}

\begin{document}
    \begin{frame}
        Let $x_{t}$ be the frequency of a strong (GC) allele at time $t$.
        \begin{itemize}
            [label=$\bullet$]
            \item We consider deleterious mutations from weak (AT) to strong (GC), with selection coefficient $s$ and dominance $h$.
            \item We consider that weak alleles are converted to strong allele in heterozygous due to gBGC with conversion rate $b$.
        \end{itemize}
        \begin{center}
            \begin{tabu}{|l||c|c|c|}
                \hline
                \textbf{Genotype}  & \textbf{Homozygous} $\bm{SS}$ & \textbf{Heterozygous} $\bm{WS}$ & \textbf{Homozygous} $\bm{WW}$ \\ \hline
                \textbf{Frequency} & $x_{t}^2$                     & $2x_{t}(1-x_{t})$               & $(1 - x_{t})^2$               \\ \hline
                \textbf{Fitness}   & $1-s$                         & $1-h s$                         & $1$                           \\ \hline
            \end{tabu}
        \end{center}
        The frequency of a strong allele at the next generation ($x_{t+1}$) is thus:
        \begin{align*}
            x_{t+1} = \frac{(1 - s)x_{t}^2 + (1 + b)(1 - h s)x_{t} (1 - x_{t})}{\bar{\omega}}.
        \end{align*}
        And the mean fitness ($\bar{\omega}$) is:
        \begin{align*}
            \bar{\omega} & = (1 - s)x_{t}^2 + (1 - h s) 2 x_{t} (1 - x_{t}) + (1 - x_{t})^2, \\
            & = 1 - s x_{t}^2 - 2 h s x_{t} (1 - x_{t}).
        \end{align*}
    \end{frame}
    \begin{frame}
        At equilibrium, the frequency of a strong allele is:
        \begin{gather*}
            x_{\text{eq}} = \frac{(1 - s) x_{\text{eq}}^2 + (1 + b)(1 - h s) x_{\text{eq}} (1 -  x_{\text{eq}})}{1 - s x_{\text{eq}}^2 - 2 h s x_{\text{eq}} (1 -  x_{\text{eq}})}, \\
            \Rightarrow x_{\text{eq}} = \frac{b-h s}{s-2h s} \text{ if } h < 1/2.
        \end{gather*}
    \end{frame}
    \begin{frame}
        Under a stochastic model, the number of strong alleles is drawn from a binomial distribution:
        \begin{align*}
            x_{t+1} & \sim \frac{1}{2 \Ne}\mathcal{B} \left( 2 \Ne, p \right), \\
            p &= \frac{(1 - s)x_{t}^2 + (1 + b)(1 - h s)x_{t} (1 - x_{t})}{\bar{\omega}}.
        \end{align*}
    \end{frame}
    \begin{frame}
        The hotspots are unstable, with lifespan $\tau$:
        \begin{gather*}
            \begin{dcases}
                t < \tau & \Rightarrow b=b_{\text{hot}}=0.02, \\
                t > \tau & \Rightarrow b=b_{\text{cold}}=6\times 10^{-6}. \\
            \end{dcases}
        \end{gather*}
    \end{frame}
    \begin{frame}
        The probability of fixation for a GC allele is $\pfix$, and the fixation load $L$ is defined as:
        \begin{equation*}
            L= s \pfix
        \end{equation*}
        \begin{equation*}
            \begin{dcases}
                L_{\tau} \text{ is the load for an unstable hotspot (lifespan $t=\tau$),} \\
                L_{\infty} \text{ is the load for a constant hotspot.}
            \end{dcases}
        \end{equation*}
    \end{frame}
\end{document}